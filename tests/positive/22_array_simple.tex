\documentclass[a4paper,11pt]{article}
\usepackage[utf8]{inputenc}
\usepackage[francais]{babel}

% --- START - HEADER TO SUPPORT TexSci ALGORITHMS IN LaTeX DOCUMENTS
\usepackage[vlined,ruled]{algorithm2e}
\usepackage{amssymb}
\usepackage{amsmath,alltt}
\newenvironment{texsci}[1]{%
\vspace{1cm}
\renewcommand{\algorithmcfname}{Algorithm}
\begin{algorithm}[H]
\label{#1}
\caption{#1}
\SetKwInOut{Constant}{Constant}
\SetKwInOut{Input}{Input}
\SetKwInOut{Output}{Output}
\SetKwInOut{Global}{Global}
\SetKwInOut{Local}{Local}
}{%
\end{algorithm}
\vspace{1cm}
}
\newcommand{\true}{\mbox{\it true}}
\newcommand{\false}{\mbox{\it false}}
\newcommand{\Boolean}{\{\true,\false\}}
\newcommand{\Integer}{\mathbb{Z}}
\newcommand{\Real}{\mathbb{R}}
% --- END - HEADER TO SUPPORT TexSci ALGORITHMS IN LaTeX DOCUMENTS

\title{TexSci Test Suite -- Test 15\\ Test tableau 1D}
\date{}

\begin{document}
\maketitle

Test des tableaux 1D. Le tableau devrait contenir les premiers termes de la suite de Fibonacci : {\tt 0 1 1 2 3 5 8 13}.

\begin{texsci}{main}
\Local{$A \in \Integer^{8},
        i \in \Integer,
        k \in \Integer$}
\BlankLine
$A_{0} \leftarrow 10$\;
$A_{1} \leftarrow 5$\;
$i \leftarrow A_{0} + A_{1}$\;
$k \leftarrow A_{0} + 3$\;

\eIf{$i = 15 \wedge k = 13$}{
  $\mbox{printInt($1$)}$\;
}{
  $\mbox{printInt($0$)}$\;
}
\end{texsci}

\end{document}
